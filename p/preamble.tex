\usepackage{comment}
\usepackage{epigraph}
\usepackage{makeidx}

\usepackage{geometry} % to change the page dimensions
%\usepackage[top=25truemm,bottom=50truemm]{geometry}



\usepackage{booktabs} % for much better looking tables
\usepackage{array} % for better arrays (eg matrices) in maths
\usepackage{paralist} % very flexible & customisable lists (eg. enumerate/itemize, etc.)
\usepackage{verbatim} % adds environment for commenting out blocks of text & for better verbatim
\usepackage{subfig} % make it possible to include more than one captioned figure/table in a single float
%\usepackage{amsmath}
\usepackage[fleqn]{amsmath}

\usepackage{amssymb}
\usepackage{braket}
\usepackage[svgnames]{xcolor}% tikzより前に読み込む必要あり
\usepackage{tikz}
\usepackage[most,listings]{tcolorbox} % for newtcbox
\usepackage{latexsym}
\usepackage{amsthm}
\usepackage{lipsum}
\usepackage{mathrsfs}
\usepackage{mathtools, nccmath}
\usepackage{xparse}
\usepackage[framemethod=tikz]{mdframed}
\usepackage{multirow}
\usepackage[normalem]{ulem}
\usepackage{tablefootnote}
\usepackage{textgreek}
\usepackage{url}
\usepackage{stackengine}
\usepackage{graphicx}
\usepackage{manfnt} %「曲がり道注意」フォント
\usepackage{relsize} %for large math
\usepackage{calc} %長さの計算のため
\usepackage{romannum} %for roman numbers \romannum{1}, \Romannum{1}, for example
\usepackage{tabularx}
\usepackage{thmtools}
\usepackage{minitoc}
\usepackage{mindflow}
\usepackage{etoolbox}
\usepackage{unicode-math}
%\setmathfont{TeX Gyre Termes Math}[
%  UppercaseGreek=Italic,
%  StylisticSet=2  % これで \mathcal が「従来風」に
%]
%\setmathfont{TeX Gyre Termes Math}[StylisticSet=1]
%\setmathfont{XITS Math}


\usepackage{unicode-math}
%\unimathsetup{math-style=TeX}
%\unimathsetup{math-style=ISO}
\setmathfont{TeX Gyre Termes Math}[UppercaseGreek=Italic]

\setmathfont{Latin Modern Math}[range=cal]

%\usepackage{fontspec}
%\usepackage[backend=biber,style=authoryear]{biblatex}
\usepackage[backend=biber,style=numeric]{biblatex}
\usepackage{ulem} % 下線を使いたい場合
\addbibresource{nsa_bibliography.bib}
%\usepackage{ntheorem}

\usetikzlibrary{patterns, intersections, calc}

\AtBeginDocument{\pagenumbering{arabic}} %ページ番号がローマ数字になるのを抑制する.
%\mathtoolsset{showonlyrefs=true}

% 引用環境を再定義
\newmdenv[
  leftline=true,          % 左の線だけ表示
  topline=false,          % 上の線なし
  bottomline=false,       % 下の線なし
  rightline=false,        % 右の線なし
  linecolor=gray,         % 線の色
  linewidth=2pt,          % 線の太さ
  innerleftmargin=10pt,   % 線と本文の間隔
  innerrightmargin=0pt,   % 右マージン
  innertopmargin=5pt,
  innerbottommargin=5pt
]{customquote}

\renewenvironment{quote}
  {\begin{customquote}\itshape}
  {\end{customquote}}


%環境
\setlength\fboxsep{7pt}
\setlength\fboxrule{0.7pt}



\NewDocumentCommand{\paren}{s m}{%
  \IfBooleanTF{#1}{%
    \mathopen{} (#2)\mathclose{}%
  }%
  {%
    \mathopen{}\left(#2 \right)\mathclose{}%
  }%
}

\NewDocumentCommand{\sqbracepair}{s m}{%
  \IfBooleanTF{#1}{%
    \mathopen{} [#2]\mathclose{}%
  }%
  {%
    \mathopen{}\left[#2 \right]\mathclose{}%
  }%
}

\NewDocumentCommand{\anglepair}{s m}{%
  \IfBooleanTF{#1}{%
    \mathopen{} \langle #2\rangle\mathclose{}%
  }%
  {%
    \mathopen{}\left\langle #2 \right\rangle\mathclose{}%
  }%
}

\NewDocumentCommand{\apply}{s m m}{%
  \IfBooleanTF{#1}{%
    #2\paren*{#3}%
  }%
  {%
    #2\paren{#3}
  }%
}

\NewDocumentCommand{\mybrace}{s m}{%
  \IfBooleanTF{#1}{%
    \mathopen{} \{#2\}\mathclose{}%
  }%
  {%
    \mathopen{}\left\lbrace{#2} \right\rbrace\mathclose{}%
  }%
}

\NewDocumentCommand{\bigparen}{m}{%
  \mathopen{}\bigl(#1 \bigr)\mathclose{}%
}

\NewDocumentCommand{\Bigparen}{m}{%
  \mathopen{}\Bigl(#1 \Bigr)\mathclose{}%
}

\NewDocumentCommand{\biggparen}{m}{%
  \mathopen{}\biggl(#1 \biggr)\mathclose{}%
}

\NewDocumentCommand{\Biggparen}{m}{%
  \mathopen{}\Biggl(#1 \Biggr)\mathclose{}%
}

\NewDocumentCommand{\offsetFromleft}{m m}{%
  {#2}\blacktriangleright{#1}
}

\NewDocumentCommand{\offsetFromRight}{m m}{%
  {#1}\blacktriangleleft{#2}
}


%■人名
\NewDocumentCommand{\PolishLos}{}{{\L}o{\'s}}
\NewDocumentCommand{\Jaskowski}{}{Ja{\'s}kowski}
\NewDocumentCommand{\PolishLukasiewicz}{}{{\L}ukasiewicz}
\NewDocumentCommand{\Hrbacek}{}{Hrb{\'a}{\v{c}}ek}
\NewDocumentCommand{\Frechet}{}{{Fr{\'e}chet}}


\DeclareMathOperator{\dom}{dom} %圏論関係
\DeclareMathOperator{\ran}{ran} %集合論関係
\DeclareMathOperator{\trcl}{trcl} %集合論,推移閉包
\DeclareMathOperator{\curry}{curry} %Haskell
\DeclareMathOperator{\uncurry}{uncurry} %Haskell
\DeclareMathOperator{\List}{List}
%\newcommand*{\ListA}{\List_{\mathrm{A}}}
%基礎集合
%\newcommand{\nullset}{\varnothing}

% 斜体ギリシア文字
%\NewDocumentCommand{\itPhi}{}{\mathit{\Phi}}
%\NewDocumentCommand{\itPsi}{}{\mathit{\Psi}}
%\NewDocumentCommand{\itTheta}{}{\mathit{\Theta}}
%\NewDocumentCommand{\itSigma}{}{\mathit{\Sigma}}

\NewDocumentCommand{\pair}{m m}{\left\langle{#1},{#2}\right\rangle}
\NewDocumentCommand{\triple}{m m m}{\left\langle{#1},\,{#2},\,{#3}\right\rangle}
\NewDocumentCommand{\quadruple}{m m m m}{\left\langle{#1},\,{#2},\,{#3},\,{#4}\right\rangle}

\NewDocumentCommand{\singleton}{m}{\left\lbrace{#1} \right\rbrace}
\NewDocumentCommand{\upair}{m m}{\left\lbrace{#1},\,{#2}\right\rbrace}
\NewDocumentCommand{\extensionalSetNotation}{m}{\left\lbrace{#1} \right\rbrace}


\NewDocumentCommand{\triff}{}{\,\triangledown\,}
\NewDocumentCommand{\shortiff}{}{\,\Leftrightarrow\,}
\NewDocumentCommand{\shortimplies}{}{\,\Rightarrow\,}
\NewDocumentCommand{\isomorphism}{}{\xrightarrow[\text{onto}]{\text{1--1}}}
\NewDocumentCommand{\ontomap}{}{\xrightarrow{\text{onto}}}
\NewDocumentCommand{\monomap}{}{\xrightarrow{\text{1--1}}}

\NewDocumentCommand{\informula}{}{$\in$-論理式\ }
\NewDocumentCommand{\instformula}{}{$\in$-$\STD$-論理式\ }


%%%%% (maybe) generic math symbols
\NewDocumentCommand{\Ker}{}{\operatorname{Ker}}
\NewDocumentCommand{\Complex}{}{\mathbb{C}}
\NewDocumentCommand{\Real}{}{\mathbb{R}}
\NewDocumentCommand{\Rational}{}{\mathbb{Q}}
\NewDocumentCommand{\Integer}{}{\mathbb{Z}}
\NewDocumentCommand{\Natural}{}{\mathbb{N}}
\NewDocumentCommand{\forany}{}{{}^\forall}
\NewDocumentCommand{\forsome}{}{{}^\exists}
\NewDocumentCommand{\True}{}{\top}
\NewDocumentCommand{\False}{}{\bot}
\NewDocumentCommand{\card}{}{\operatorname{card}}
\NewDocumentCommand{\cardArg}{m}{\card\mathopen{}\left({#1} \right)\mathclose{}}


\NewDocumentCommand{\Map}{}{\operatorname{Map}}
\NewDocumentCommand{\Mapop}{m m}{\apply{\Map}{#1, #2}}

\NewDocumentCommand{\FMap}{}{\operatorname{FMap}}
\NewDocumentCommand{\FMapop}{m m}{\apply{\FMap}{#1, #2}}

\NewDocumentCommand{\Fun}{}{\mathbf{Fun}}
\NewDocumentCommand{\FFun}{}{\mathbf{FFun}}


\NewDocumentCommand{\Mat}{}{\operatorname{Mat}}
\NewDocumentCommand{\Seq}{}{\operatorname{Seq}}
\NewDocumentCommand{\superstructure}{m}{\widehat{{#1}}}
\NewDocumentCommand{\superS}{}{\superstructure{S}}


\NewDocumentCommand{\starmap}{m}{\prescript{*}{}{#1}}

\NewDocumentCommand{\hyperReal}{}{\starmap{\Real}}


\NewDocumentCommand{\ul}{m}{\underline{#1}}


%■重要なクラス

\NewDocumentCommand{\Ord}{}{\mathbf{Ord}}
\NewDocumentCommand{\NatInfty}{}{\Nat_{\infty}}
\NewDocumentCommand{\RealLimited}{}{\Real_{\mathrm{limited}}}

\RenewDocumentCommand{\bar}{m}{\,\overline{#1}\,}
\NewDocumentCommand{\manyvar}{m}{\kern.11em\overline{#1}\kern.11em}

\NewDocumentCommand{\Fin}{}{\mathrm{Fin}}






\NewDocumentCommand\standarization{m m}{\prescript{\STD}{}{\set{#1}{#2}}}

%\newcommand{\stpart}[1]{{}^\circ{#1}}
\NewDocumentCommand\stpart{m}{{}^\circ{#1}}


\NewDocumentCommand{\ZFC}{}{\textbf{ZFC}}
\NewDocumentCommand{\IST}{}{\textbf{IST}}
\NewDocumentCommand{\BST}{}{\textbf{BST}}
\NewDocumentCommand{\RBST}{}{\textbf{RBST}}
\NewDocumentCommand{\KST}{}{\textbf{KST}}
\NewDocumentCommand{\HST}{}{\textbf{HST}}

\makeatletter
\newcommand{\leqnomode}{\tagsleft@true}
\newcommand{\reqnomode}{\tagsleft@false}
\makeatother


%■議論■
\NewDocumentCommand{\iffdef}{}{\mathrel{\overset{\makebox[0pt]{\mbox{\normalfont\tiny\sffamily def}}}{\longleftrightarrow}}}

\NewDocumentCommand{\lequiv}{}{\enskip\longleftrightarrow\enskip}

%■集合論補助■
%\NewDocumentCommand{\powerset}{m}{\mathcal{P}\mathopen{}\left({#1}\right)\mathclose{}}
%\NewDocumentCommand{\powersetfin}{m}{\mathcal{P}_{\mathrm{Fin}}\mathopen{}\left({#1}\right)\mathclose{}}

\NewDocumentCommand{\powerset}{m}{\apply{\mathcal{P}}{#1}}
\NewDocumentCommand{\powersetfin}{m}{\apply{\mathcal{P}_{\mathrm{Fin}}}{#1}}


%\newcommand{\powersetfin}[1]{\mathcal{P}_{\mathrm{Fin}}\mathopen{}\left({#1}\right)\mathclose{}}
\NewDocumentCommand{\Nat}{}{\mathbb{N}}
\NewDocumentCommand{\plusone}{m}{{#1} \cup\singleton{{#1}}}

\NewDocumentCommand{\prednamestd}{}{\operatorname{st}}
\NewDocumentCommand{\std}{m}{\apply{\prednamestd}{#1}}
\NewDocumentCommand{\STD}{}{\mathbf{S}}
\NewDocumentCommand{\relativeizationS}{m}{{#1}^{\STD}}

\NewDocumentCommand{\metaquote}{m}{\text{"${#1}$"}}
\NewDocumentCommand{\lparenitself}{}{(}
\NewDocumentCommand{\rparenitself}{}{)}

\NewDocumentCommand{\forallfin}{}{\forall^{\Fin}}
\NewDocumentCommand{\existsfin}{}{\exists^{\Fin}}

%------------------------------------------------
\NewDocumentCommand{\forallst}{}{\forall^\STD}
\NewDocumentCommand{\existsst}{}{\exists^\STD}
\NewDocumentCommand{\existsstfin}{}{\exists^{\STD\,\Fin}}
\NewDocumentCommand{\forallstfin}{}{\forall^{\STD\,\Fin}}
%------------------------------------------------
\NewDocumentCommand{\RelStd}{m}{\STD\sqbracepair*{#1}}
%------------------------------------------------
\NewDocumentCommand{\forallRelst}{m}{\forall^{\RelStd{#1}}}
\NewDocumentCommand{\existsRelst}{m}{\exists^{\RelStd{#1}}}
\NewDocumentCommand{\existsRelstfin}{m}{\exists^{\RelStd{#1}\,\Fin}}
\NewDocumentCommand{\forallRelstfin}{m}{\forall^{\RelStd{#1}\,\Fin}}


%------------------------------------------------
\NewDocumentCommand{\WeakRelStd}{m}{\STD\anglepair*{#1}}
%------------------------------------------------
\NewDocumentCommand{\forallWeakRelst}{m}{\forall^{\weakRelStd{#1}}}
\NewDocumentCommand{\existsWeakRelst}{m}{\exists^{\weakRelStd{#1}}}
\NewDocumentCommand{\existsWeakRelstfin}{m}{\exists^{\weakRelStd{#1}\,\Fin}}
\NewDocumentCommand{\forallWeakRelstfin}{m}{\forall^{\weakRelStd{#1}\,\Fin}}
%---------------


\NewDocumentCommand{\nullset}{}{\varnothing}
\NewDocumentCommand{\TheUniverse}{}{\mathbf{V}}


\NewDocumentCommand{\RealPlus}{}{\Real_{+}}


\NewDocumentCommand{\hal}{}{\mathrm{hal}}
\NewDocumentCommand{\gal}{}{\mathrm{gal}}

\NewDocumentCommand{\ball}{}{\mathrm{Ball}}


%集合マクロ
\RenewDocumentCommand{\set}{m m}{\left\lbrace{#1}\:\,\vrule\:\,{#2}\right\rbrace}
%\renewcommand{\set}[2]{\left\lbrace{#1}\:\,\vrule\:\,{#2}\right\rbrace}

\newcommand*{\mat}[5]{\left[ {#1} \,\vrule\:\
\begin{array}{ccccc}																%
      {#2}  &\kern-.7em\colon &\kern-.7em 1  &\kern-.7em \downarrow &\kern-.7em {#4} \\ %
      {#3}  &\kern-.7em\colon &\kern-.7em 1  &\kern-.7em \to        &\kern-.7em {#5}     %
\end{array}
 \right]}



%%%定理カウンタ
\newcounter{mycounter} % カウンタの宣言
\setcounter{mycounter}{0} % カウンタの初期化
\newcommand{\useMycounter}[1][]{\refstepcounter{mycounter}{#1}{\themycounter}\:}

%%%定義カウンタ
\newcounter{defcounter}
\setcounter{defcounter}{0}
\newcommand{\useDefcounter}[1][]{\refstepcounter{defcounter}{#1}{\thedefcounter}\:}



\NewDocumentCommand{\bendwarning}{}{\raisebox{1.5ex}{\dbend}}%「曲道注意」を入れる.


%%% xor 記号 %%%
\NewDocumentCommand{\barbelow}{m}{\stackunder[1.2pt]{$#1$}{\rule{.8ex}{.075ex}}}
\NewDocumentCommand{\xor}{}{\,\barbelow{\lor}\,}

%%% concat %%%
\NewDocumentCommand{\verysmallnegativespace}{}{\kern-0.16em}
\NewDocumentCommand{\concat}{}{\verysmallnegativespace::\verysmallnegativespace}


\newtcbox{\frameforimplication}{
    on line, 
    arc=2.1pt,
    enhanced jigsaw,
    opacityback = 0,
    colframe = black,
    %before upper={\rule[-3pt]{0pt}{10pt}},
    before = {},
    after = {},
    boxrule=1pt,
    boxsep=1.3pt,
    left=3.5pt,
    right=3.5pt,
    top=2.6pt,
    bottom=1.9pt
}

\newtcbox{\framedtext}{
    on line, 
    arc=2.1pt,
    enhanced jigsaw,
    opacityback = 0,
    colframe = black,
    %before upper={\rule[-3pt]{0pt}{10pt}},
    before = {},
    after = {},
    boxrule=1pt,
    boxsep=1.3pt,
    left=3.5pt,
    right=3.5pt,
    top=0.3pt,
    bottom=0.3pt
}






\NewDocumentCommand{\casesplit}{m}{\noindent\framedtext{{#1}}\quad}

\NewDocumentCommand{\labeltext}{m}{\noindent\framedtext{{#1}}\quad}

\NewDocumentCommand{\labeledarrow}{m}{\noindent\frameforimplication{#1}\enskip}

\NewDocumentCommand{\labeledRightarrow}{}{\labeledarrow{$\Rightarrow$}}

\NewDocumentCommand{\labeledLeftarrow}{}{\labeledarrow{$\Leftarrow$}}



\NewDocumentCommand{\opK}{m m}{\operatorname{\mathsf{K}}\mathopen{}\left({#1},{#2} \right)\mathclose{}}


\NewDocumentCommand{\AxBoundedness}{}{\textsf{Boundedness}}
\NewDocumentCommand{\AxTransfer}{}{\textsf{Transfer}}
\NewDocumentCommand{\AxStandarization}{}{\textsf{Standarization}}
\NewDocumentCommand{\AxBoundedIdealization}{}{\textsf{Bounded Idealization}}
\NewDocumentCommand{\AxIdealization}{}{\textsf{Idealization}}
\NewDocumentCommand{\PropLocalIdealization}{}{\textsf{Local Idealization}}

\NewDocumentCommand{\AxBbare}{}{\textsf{B}}
\NewDocumentCommand{\AxTbare}{}{\textsf{T}}
\NewDocumentCommand{\AxSbare}{}{\textsf{S}}
\NewDocumentCommand{\AxBIbare}{}{\textsf{BI}}
\NewDocumentCommand{\AxIbare}{}{\textsf{I}}

\NewDocumentCommand{\AxB}{}{(\AxBbare)}
\NewDocumentCommand{\AxT}{}{(\AxTbare)}
\NewDocumentCommand{\AxS}{}{(\AxSbare)}
\NewDocumentCommand{\AxBI}{}{(\AxBIbare)}
\NewDocumentCommand{\AxI}{}{(\AxIbare)}

\NewDocumentCommand{\PropLI}{}{\textsf{(LI)}}



\makeatletter
\renewcommand*{\thefootnote}{%
  \arabic{footnote}%
}
\makeatother

%■mdframed の問題を修正
\makeatletter
\AfterEndEnvironment{mdframed}{%
 \tfn@tablefootnoteprintout%
 \gdef\tfn@fnt{0}%
}%
\makeatother

\setlength{\epigraphwidth}{0.85\linewidth}


% proof環境のカスタマイズ
\makeatletter
\AtBeginDocument{%
  \renewcommand\proofname{\textbf{証明}}% 証明ラベルを太字に
  \let\oldproof\proof % 元のproof環境を保存
  \renewcommand{\proof}[1][\proofname]{% proof環境を再定義
    \par
    \pushQED{\qed}%
    \normalfont \topsep6\p@\@plus6\p@\relax
    \trivlist
    \item[\hskip\labelsep
          \textbf{#1}]% カスタムタイトルも太字に
    \ignorespaces
  }
}
\makeatother

\makeatletter%
\let\@addpunct\@gobble%
\makeatother%
% ↑この設定によって\bf{証明}の後の「.」が落ちる。

\mdfsetup{%
   middlelinewidth=0.9pt,
   roundcorner=4.3pt,
   innertopmargin=1.5pt,
   innerbottommargin=7.3pt, 
}

%%% 定理環境
\declaretheoremstyle[
  headfont=\normalfont\bfseries,
  bodyfont=\normalfont,
  headpunct={},
  postheadspace=0.8em,
  mdframed={},
]{framedstyle}

\newcommand{\thmstart}{\leavevmode\strut\par}

%%========================
%% 定義
%%========================
\declaretheorem[
  numberwithin=section,
  style=framedstyle,
  name={定義}
]{definition}


%%========================
%% 定理
%%========================
\declaretheorem[
  numberwithin=section,
  style=framedstyle,
  name={定理}
]{theorem}


%%========================
%% 命題
%%========================
\declaretheorem[
  numberwithin=section,
  style=framedstyle,
  sibling=theorem,
  name={命題}
]{prop}


%%========================
%% 補題
%%========================
\declaretheorem[
  numberwithin=section,
  style=framedstyle,
  sibling=theorem,
  name={補題}
]{lemma}


%%========================
%% 系
%%========================
\declaretheorem[
  numberwithin=section,
  style=framedstyle,
  sibling=theorem,
  name={系}
]{corollary}


%%========================
%% 記法
%%========================
\declaretheorem[
  numbered=no,
  style=framedstyle,
  name={記法}
]{notation}


%%========================
%% ノート
%%========================
\declaretheorem[
  numbered=no,
  style=framedstyle,
  name={ノート}
]{note}


%%========================
%% 注意
%%========================
\declaretheorem[
  numbered=no,
  style=framedstyle,
  name={注意}
]{caution}

%%========================
%% 例
%%========================
%\mdfsetup{backgroundcolor=orange!20}
\declaretheorem[
  numberwithin=section,
  style=framedstyle,
  name={例}
]{example}


%%========================
%% (ただの枠)
%%========================
\newmdenv[
  middlelinewidth=0.9pt,
  roundcorner=4.3pt,
  innertopmargin=6.4pt,
  innerbottommargin=6.4pt,
  skipabove=9pt,
  skipbelow=3.5pt
]{screen}

\newmdenv[
  middlelinewidth=0.9pt,
  roundcorner=4.3pt,
  innertopmargin=-6.2pt,
  innerbottommargin=6.4pt,
  skipabove=9pt,
  skipbelow=3.5pt
]{single-math-screen}


%%========================
%% 赤枠
%%========================
\declaretheorem[
  numbered=no,
  style=redframestyle,
  name={},
]{redscreen}


%emphasizestyle
%== blanket argument
\declaretheorem[
  numbered=no,
  style=emphasizestyle,
  name={規約}
]{convention}


%%========================
%% (ただの枠)
%%========================
\declaretheorem[
    %headindent = 0,
    numbered=no,
    style=framedstyle,
    name={}
]{justbox}


%%==============================
%%  定理(チートシート用)
%%==============================
\declaretheorem[
    numbered=no,
    style=framedstyle,
    name={Theorem}
]{cheatsheetthm}


%%==============================
%%  命題(チートシート用)
%%==============================
\declaretheorem[
    numbered=no,
    style=framedstyle,
    name={Proposition}
]{cheatsheetprop}


\RenewDocumentCommand{\emph}{m d()}{%
  \IfValueTF{#2}{%
    \textbf{#1}~(\textit{#2})%
  }%
  {%
    \textbf{#1}%
  }%
}



%文献名表示フック
\renewrobustcmd*{\mkbibemph}[1]{\textit{#1}}
%\DeclareFieldFormat[english]{title}{\mkbibemph{#1}} % 英語のときはイタリック
%\DeclareFieldFormat[japanese]{title}{\textbf{#1}}   % 日本語のときは太字


\geometry{a4paper} % or letterpaper (US) or a5paper or....
