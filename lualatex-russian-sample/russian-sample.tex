% Sample Document for LuaLaTeX, with Russian Language.

% To compile: just run $ lualatex russian-sample.tex

\documentclass[12pt]{article}
\usepackage{amsmath}
\usepackage{latexsym}
\usepackage{amsfonts}
\usepackage{amssymb}
\usepackage{polyglossia}
\usepackage{fontspec}

% "---" should be emdash
\defaultfontfeatures{Scale=MatchLowercase,Ligatures=TeX}
\setdefaultlanguage{russian}
\newfontfamily\cyrillicfont{CMU Serif}
\setmainlanguage{russian}

\begin{document}

Одиако при $\alpha \sim 0 \quad f(\alpha, y)$ --- уже нестапдартная функция и эквивалеитность (1) гі ней не применима. 
Этот пример наводит на мысль о необходимости введения бесконечно малых \textit{существенно более высокого порядка, чем даниое} $\alpha$, т. е. таких, ноторые остаются бесконечшо малыми, даже если считать $\alpha$ конечным. 
Такая мысыь высказана в 70-е годы А. Г. Драгалиным на семинарах в Московском государственном уииверситете. 

\end{document}