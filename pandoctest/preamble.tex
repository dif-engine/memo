\usepackage{comment}
\usepackage{epigraph}
\usepackage{makeidx}

\usepackage{geometry} % to change the page dimensions
%\usepackage[top=25truemm,bottom=50truemm]{geometry}

\usepackage{booktabs} % for much better looking tables
\usepackage{array} % for better arrays (eg matrices) in maths
\usepackage{paralist} % very flexible & customisable lists (eg. enumerate/itemize, etc.)
\usepackage{verbatim} % adds environment for commenting out blocks of text & for better verbatim
\usepackage{subfig} % make it possible to include more than one captioned figure/table in a single float
\usepackage{amsmath}
\usepackage{amssymb}
\usepackage{braket}
\usepackage[svgnames]{xcolor}% tikzより前に読み込む必要あり
\usepackage{tikz}
\usepackage[most,listings]{tcolorbox} % for newtcbox
\usepackage{latexsym}
\usepackage{amsthm}
\usepackage{lipsum}
\usepackage{mathrsfs}
\usepackage{mathtools, nccmath}
\usepackage{xparse}
\usepackage[framemethod=tikz]{mdframed}
\usepackage{multirow}
\usepackage[normalem]{ulem}
\usepackage{tablefootnote}
\usepackage{textgreek}
\usepackage{url}
\usepackage{stackengine}
\usepackage{graphicx}
\usepackage{manfnt} %「曲がり道注意」フォント
\usepackage{relsize} %for large math
\usepackage{calc} %長さの計算のため
\usepackage{romannum} %for roman numbers \romannum{1}, \Romannum{1}, for example
\usepackage{tabularx}
\usepackage{thmtools}
\usepackage{minitoc}
\usepackage{mindflow}
\usepackage{etoolbox}

\usetikzlibrary{patterns, intersections, calc}

\AtBeginDocument{\pagenumbering{arabic}} %ページ番号がローマ数字になるのを抑制する.


%===============================================
% カッコ関係
%===============================================

\NewDocumentCommand{\paren}{s m}{%
  \IfBooleanTF{#1}{%
    \mathopen{} (#2)\mathclose{}%
  }%
  {%
    \mathopen{}\left(#2 \right)\mathclose{}%
  }%
}

\NewDocumentCommand{\sqbracepair}{s m}{%
  \IfBooleanTF{#1}{%
    \mathopen{} [#2]\mathclose{}%
  }%
  {%
    \mathopen{}\left[#2 \right]\mathclose{}%
  }%
}

\NewDocumentCommand{\anglepair}{s m}{%
  \IfBooleanTF{#1}{%
    \mathopen{} \langle #2\rangle\mathclose{}%
  }%
  {%
    \mathopen{}\left\langle #2 \right\rangle\mathclose{}%
  }%
}

\NewDocumentCommand{\apply}{s m m}{%
  \IfBooleanTF{#1}{%
    #2\paren*{#3}%
  }%
  {%
    #2\paren{#3}
  }%
}

\NewDocumentCommand{\mybrace}{s m}{%
  \IfBooleanTF{#1}{%
    \mathopen{} \{#2\}\mathclose{}%
  }%
  {%
    \mathopen{}\left\lbrace{#2} \right\rbrace\mathclose{}%
  }%
}

\NewDocumentCommand{\bigparen}{m}{%
  \mathopen{}\bigl(#1 \bigr)\mathclose{}%
}

\NewDocumentCommand{\Bigparen}{m}{%
  \mathopen{}\Bigl(#1 \Bigr)\mathclose{}%
}

\NewDocumentCommand{\biggparen}{m}{%
  \mathopen{}\biggl(#1 \biggr)\mathclose{}%
}

\NewDocumentCommand{\Biggparen}{m}{%
  \mathopen{}\Biggl(#1 \Biggr)\mathclose{}%
}

\NewDocumentCommand{\offsetFromleft}{m m}{%
  {#2}\blacktriangleright{#1}
}

\NewDocumentCommand{\offsetFromRight}{m m}{%
  {#1}\blacktriangleleft{#2}
}

%===============================================
% 集合論に関連した記法
%===============================================
\RenewDocumentCommand{\set}{m m}{\left\lbrace{#1}\:\,\vrule\:\,{#2}\right\rbrace}
\NewDocumentCommand{\powerset}{m}{\mathcal{P}\mathopen{}\left({#1}\right)\mathclose{}}
\NewDocumentCommand{\pair}{m m}{\left\langle{#1},\,{#2}\right\rangle}
\NewDocumentCommand{\triple}{m m m}{\left\langle{#1},\,{#2},\,{#3}\right\rangle}
\NewDocumentCommand{\quadruple}{m m m m}{\left\langle{#1},\,{#2},\,{#3},\,{#4}\right\rangle}

\NewDocumentCommand{\singleton}{m}{\left\lbrace{#1} \right\rbrace}
\NewDocumentCommand{\upair}{m m}{\left\lbrace{#1},\,{#2}\right\rbrace}
\NewDocumentCommand{\extensionalSetNotation}{m}{\left\lbrace{#1} \right\rbrace}

\NewDocumentCommand{\isomorphism}{}{\xrightarrow[\text{onto}]{\text{1--1}}}
\NewDocumentCommand{\ontomap}{}{\xrightarrow{\text{onto}}}
\NewDocumentCommand{\monomap}{}{\xrightarrow{\text{1--1}}}

%===============================================
% 良く議論で使うものたち
%===============================================
\NewDocumentCommand{\bendwarning}{}{\raisebox{1.5ex}{\dbend}}%「曲道注意」を入れる.
\NewDocumentCommand{\Complex}{}{\mathbb{C}}
\NewDocumentCommand{\Real}{}{\mathbb{R}}
\NewDocumentCommand{\Rational}{}{\mathbb{Q}}
\NewDocumentCommand{\Integer}{}{\mathbb{Z}}
\NewDocumentCommand{\Nat}{}{\mathbb{N}}
\NewDocumentCommand{\True}{}{\top}
\NewDocumentCommand{\False}{}{\bot}
\NewDocumentCommand{\iffdef}{}{\mathrel{\overset{\makebox[0pt]{\mbox{\normalfont\tiny\sffamily def}}}{\longleftrightarrow}}}
\NewDocumentCommand{\lequiv}{}{\enskip\longleftrightarrow\enskip}

%===============================================
% 数式番号の左右の変更
%===============================================
\makeatletter
\newcommand{\leqnomode}{\tagsleft@true}
\newcommand{\reqnomode}{\tagsleft@false}
\makeatother


%===============================================
% 各種カウンタの設定
%===============================================

%定理カウンタ
\newcounter{mycounter} % カウンタの宣言
\setcounter{mycounter}{0} % カウンタの初期化
\newcommand{\useMycounter}[1][]{\refstepcounter{mycounter}{#1}{\themycounter}\:}

%定義カウンタ
\newcounter{defcounter}
\setcounter{defcounter}{0}
\newcommand{\useDefcounter}[1][]{\refstepcounter{defcounter}{#1}{\thedefcounter}\:}


%===============================================
% 枠付きボックスのレンダリング
%===============================================
\setlength\fboxsep{7pt}
\setlength\fboxrule{0.7pt}

\newtcbox{\framedtext}{
    on line, 
    arc=2.1pt,
    enhanced jigsaw,
    opacityback = 0,
    colframe = black,
    %before upper={\rule[-3pt]{0pt}{10pt}},
    before = {},
    after = {},
    boxrule=1pt,
    boxsep=1.3pt,
    left=4pt,
    right=4pt,
    top=2pt,
    bottom=0.5pt
}

%===============================================
% 脚注番号をアラビア数字に
%===============================================
\makeatletter
\renewcommand*{\thefootnote}{%
  \arabic{footnote}%
}
\makeatother

%===============================================
% mdframed環境の中でも脚注が正常に使えるようにする
%===============================================
\makeatletter
\AfterEndEnvironment{mdframed}{%
 \tfn@tablefootnoteprintout%
 \gdef\tfn@fnt{0}%
}%
\makeatother

%===============================================
% エピグラフの長さの調整
%===============================================
\setlength{\epigraphwidth}{1.2\linewidth}

%===============================================
% 証明環境のカスタマイズ(1)
%===============================================
\makeatletter
\AtBeginDocument{%
  \renewcommand\proofname{\textbf{証明}}% 証明ラベルを太字に
  \let\oldproof\proof % 元のproof環境を保存
  \renewcommand{\proof}[1][\proofname]{% proof環境を再定義
    \par
    \pushQED{\qed}%
    \normalfont \topsep6\p@\@plus6\p@\relax
    \trivlist
    \item[\hskip\labelsep
          \textbf{#1}]% カスタムタイトルも太字に
    \ignorespaces
  }
}
\makeatother

%===============================================
% 証明環境のカスタマイズ(2)
%===============================================
\makeatletter%
\let\@addpunct\@gobble%
\makeatother%
% ↑この設定によって「証明」の後の「.」が落ちる。

%===============================================
% mdframed 環境のデフォルト値の設定
%===============================================
\mdfsetup{%
   middlelinewidth=0.9pt,
   roundcorner=4.3pt,
   innertopmargin=7pt,
   innerbottommargin=7pt, 
   skipabove=9pt, 
   skipbelow=9pt,
}

%%% 定理環境
\declaretheoremstyle[
    headfont=\normalfont\bfseries, 
    notefont=\normalfont\bfseries,
    bodyfont=\normalfont, %not italic
    headpunct={},
    postheadspace=8pt,
    spaceabove=\itemsep,
    spacebelow=\itemsep,
    mdframed={}
]{framedstyle}

\declaretheoremstyle[
    headfont=\normalfont\bfseries, 
    notefont=\normalfont\bfseries,
    bodyfont=\normalfont, %not italic
    headpunct={},
    postheadspace=8pt,
    spaceabove=\itemsep,
    spacebelow=\itemsep,
    mdframed={
    middlelinewidth=2.7pt,
    middlelinecolor=red!70!black,
   },
]{redframestyle}

%===============================================
% 環境開始直後かどうかを判定するフラグの導入
%===============================================
\newif\ifAtThmEnvStart
\AtThmEnvStartfalse % デフォルトではフラグをfalseに

%%========================
%% 定義
%%========================
\declaretheorem[
  numberwithin=section,
  style=framedstyle,
  name={定義}
]{definition}
% 環境の開始時にフラグをtrueに設定
\AtBeginEnvironment{definition}{
  \AtThmEnvStarttrue
  \everypar{\AtThmEnvStartfalse\everypar{}}
}

%%========================
%% 定理
%%========================
\declaretheorem[
  numberwithin=section,
  style=framedstyle,
  name={定理}
]{theorem}
% 環境の開始時にフラグをtrueに設定
\AtBeginEnvironment{theorem}{
  \AtThmEnvStarttrue
  \everypar{\AtThmEnvStartfalse\everypar{}}
}

%%========================
%% 命題
%%========================
\declaretheorem[
  numberwithin=section,
  style=framedstyle,
  sibling=theorem,
  name={命題}
]{prop}
% 環境の開始時にフラグをtrueに設定
\AtBeginEnvironment{prop}{
  \AtThmEnvStarttrue
  \everypar{\AtThmEnvStartfalse\everypar{}}
}

%%========================
%% 補題
%%========================
\declaretheorem[
  numberwithin=section,
  style=framedstyle,
  sibling=theorem,
  name={補題}
]{lemma}
% 環境の開始時にフラグをtrueに設定
\AtBeginEnvironment{lemma}{
  \AtThmEnvStarttrue
  \everypar{\AtThmEnvStartfalse\everypar{}}
}

%%========================
%% 系
%%========================
\declaretheorem[
  numberwithin=section,
  style=framedstyle,
  sibling=theorem,
  name={系}
]{corollary}
% 環境の開始時にフラグをtrueに設定
\AtBeginEnvironment{corollary}{
  \AtThmEnvStarttrue
  \everypar{\AtThmEnvStartfalse\everypar{}}
}

%%========================
%% 記法
%%========================
\declaretheorem[
  numbered=no,
  style=framedstyle,
  name={記法}
]{notation}
% 環境の開始時にフラグをtrueに設定
\AtBeginEnvironment{notation}{
  \AtThmEnvStarttrue
  \everypar{\AtThmEnvStartfalse\everypar{}}
}

%%========================
%% ノート
%%========================
\declaretheorem[
  numbered=no,
  style=framedstyle,
  name={ノート}
]{note}
% 環境の開始時にフラグをtrueに設定
\AtBeginEnvironment{note}{
  \AtThmEnvStarttrue
  \everypar{\AtThmEnvStartfalse\everypar{}}
}

%%========================
%% 注意
%%========================
\declaretheorem[
  numbered=no,
  style=framedstyle,
  name={注意}
]{caution}
% 環境の開始時にフラグをtrueに設定
\AtBeginEnvironment{caution}{
  \AtThmEnvStarttrue
  \everypar{\AtThmEnvStartfalse\everypar{}}
}

%%========================
%% 例
%%========================
%\mdfsetup{backgroundcolor=orange!20}
\declaretheorem[
  numberwithin=section,
  style=framedstyle,
  name={例}
]{example}
% 環境の開始時にフラグをtrueに設定
\AtBeginEnvironment{example}{
  \AtThmEnvStarttrue
  \everypar{\AtThmEnvStartfalse\everypar{}}
}

%%========================
%% (ただの枠)
%%========================
\declaretheorem[
  numbered=no,
  style=framedstyle,
  name={}
]{screen}
% 環境の開始時にフラグをtrueに設定
\AtBeginEnvironment{screen}{
  \AtThmEnvStarttrue
  \everypar{\AtThmEnvStartfalse\everypar{}}
}

%%========================
%% 赤枠
%%========================
\declaretheorem[
  numbered=no,
  style=redframestyle,
  name={},
]{redscreen}
% 環境の開始時にフラグをtrueに設定
\AtBeginEnvironment{redscreen}{
  \AtThmEnvStarttrue
  \everypar{\AtThmEnvStartfalse\everypar{}}
}

%%========================
%% (ただの枠)
%%========================
\declaretheorem[
    %headindent = 0,
    numbered=no,
    style=framedstyle,
    name={}
]{justbox}
% 環境の開始時にフラグをtrueに設定
\AtBeginEnvironment{justbox}{
  \AtThmEnvStarttrue
  \everypar{\AtThmEnvStartfalse\everypar{}}
}

%%==============================
%%  定理(チートシート用)
%%==============================
\declaretheorem[
    numbered=no,
    style=framedstyle,
    name={Theorem}
]{cheatsheetthm}
% 環境の開始時にフラグをtrueに設定
\AtBeginEnvironment{cheatsheetthm}{
  \AtThmEnvStarttrue
  \everypar{\AtThmEnvStartfalse\everypar{}}
}

%%==============================
%%  命題(チートシート用)
%%==============================
\declaretheorem[
    numbered=no,
    style=framedstyle,
    name={Proposition}
]{cheatsheetprop}
% 環境の開始時にフラグをtrueに設定
\AtBeginEnvironment{cheatsheetprop}{
  \AtThmEnvStarttrue
  \everypar{\AtThmEnvStartfalse\everypar{}}
}


%%%%%%%%%%%%%%%%%%%%%%%%%%%%%%%%%%%%%%%%%%%%%%%%%%%%%%%%%%%
%%%%%%%%%%%%%%%%%%%%%%%%%%%%%%%%%%%%%%%%%%%%%%%%%%%%%%%%%%%

\newlength{\defaultabovedisplayskip}
\setlength{\defaultabovedisplayskip}{\abovedisplayskip}

% align環境の開始時にフラグをチェックし、間隔を設定
\BeforeBeginEnvironment{align}{%
  \ifAtThmEnvStart
    \setlength{\abovedisplayskip}{0pt}% 環境開始直後の場合、間隔を0ptに
    \AtThmEnvStartfalse % フラグをfalseに(次の数式では通常の間隔を使用)
  \else
    \setlength{\abovedisplayskip}{\defaultabovedisplayskip}% それ以外の場合はデフォルト(例: 10pt)
  \fi
}

% align* 環境の開始時にフラグをチェックし、間隔を設定
\BeforeBeginEnvironment{align*}{%
  \ifAtThmEnvStart
    \setlength{\abovedisplayskip}{0pt}% 環境開始直後の場合、間隔を0ptに
    \AtThmEnvStartfalse % フラグをfalseに(次の数式では通常の間隔を使用)
  \else
    \setlength{\abovedisplayskip}{\defaultabovedisplayskip}% それ以外の場合はデフォルト(例: 10pt)
  \fi
}

% equation 環境の開始時にフラグをチェックし、間隔を設定
\BeforeBeginEnvironment{equation}{%
  \ifAtThmEnvStart
    \setlength{\abovedisplayskip}{0pt}%
    \AtThmEnvStartfalse
  \else
    \setlength{\abovedisplayskip}{\defaultabovedisplayskip}%
  \fi
}

% equation*環境の開始時にフラグをチェックし、間隔を設定
\BeforeBeginEnvironment{equation*}{%
  \ifAtThmEnvStart
    \setlength{\abovedisplayskip}{0pt}%
    \AtThmEnvStartfalse
  \else
    \setlength{\abovedisplayskip}{\defaultabovedisplayskip}%
  \fi
}


\RenewDocumentCommand{\emph}{m d()}{%
  \IfValueTF{#2}{%
    \textbf{#1}~(\textit{#2})%
  }%
  {%
    \textbf{#1}%
  }%
}



